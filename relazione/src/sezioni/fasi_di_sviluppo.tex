\section{Fasi di Sviluppo}
\subsection{Sviluppo generale}
Dopo esserci consultati e accordati sul tema da sviluppare, abbiamo iniziato a generare le sei pagine base in HTML sistemandole graficamente attraverso fogli di stile CSS. Ottenendo un sito statico soddisfacente, abbiamo creato il database, e abbiamo iniziato a sviluppare le funzioni in linguaggio PHP, adattando il lavoro già svolto al fine di dare al sito una risposta dinamica e limitare la riscrittura manuale di codice. Ciò ha ridotto di molto il lavoro necessario per alcuni aspetti, come la stesura e la successiva modifica degli HTML ripetitivi. Poi, per permettere il controllo sugli input, sono state inserite ulteriori funzioni tramite il linguaggio JavaScript. Infine, quando il sito è stato considerato completo, sono stati eseguiti gli ultimi controlli e test.
\subsection{Divisione dei compiti}
Abbiamo deciso di dividere il lavoro in pagine e funzionalità piuttosto che dividerle per linguaggio in modo da permettere a ciascun membro del gruppo di prendere confidenza con tutti i linguaggi e per poter dare supporto agli altri in caso di necessità, ma si è rivelata una brutta scelta in corso d’opera. Da metà progetto in poi, infatti, la divisione è stata per ambiti, orizzontalmente. 
Nello specifico:\\

Enrico si è occupato del PHP legato alla generazione dinamica, php menu, pagine home, parte privata, del CSS per le pagine home e la parte privata.\\

Tobia invece si è occupato delle pagine HTML e del relativo CSS: attorno a noi , i nostri servizi, contattaci e prenota. E si è occupato anche del CSS per la parte mobile del sito.\\

Francesco si è occupato delle pagine HTML e del relativo CSS: come raggiungerci e le nostre camere. CSS di stampa con Silvia e ha scritto la relazione con l'aiuto di Tobia.\\

Silvia si è occupata come già detto: della stampa CSS, della stesura della relazione con Francesco e dei controlli lato client in Javascript. 