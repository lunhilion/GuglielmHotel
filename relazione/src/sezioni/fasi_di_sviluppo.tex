\section{Fasi di Sviluppo}
\subsection{Sviluppo generale}
Dopo esserci consultati e accordati sul tema da sviluppare, abbiamo iniziato a generare le sei pagine base in HTML sistemandole graficamente attraverso fogli di stile CSS. Ottenendo un sito statico soddisfacente, abbiamo creato il database, e abbiamo iniziato a sviluppare le funzioni in linguaggio PHP, adattando il lavoro già svolto al fine di dare al sito una risposta dinamica e limitare la riscrittura manuale di codice. Ciò ha ridotto di molto il lavoro necessario per alcuni aspetti, come la stesura e la successiva modifica degli HTML ripetitivi. Poi, per permettere il controllo sugli input, sono state inserite ulteriori funzioni tramite il linguaggio JavaScript. Infine, quando il sito è stato considerato completo, sono stati eseguiti gli ultimi controlli e test.
\subsection{Divisione dei compiti}
Abbiamo deciso di dividere il lavoro in pagine e funzionalità piuttosto che dividerle per linguaggio in modo da permettere a ciascun membro del gruppo di prendere confidenza con tutti i linguaggi e per poter dare supporto agli altri in caso di necessità, ma si è rivelata una brutta scelta in corso d’opera. Da metà progetto in poi, infatti, la divisione è stata per ambiti, orizzontalmente. 
Nello specifico:\\

Enrico si è occupato del PHP legato alla generazione dinamica delle pagine, del CSS di alcune pagine e della validazione.\\

Tobia invece si è occupato delle pagine HTML e del relativo CSS: attorno a noi , i nostri servizi, contattaci e prenota. Si è occupato, inoltre, anche del CSS per la parte mobile del sito.\\

Francesco si è occupato dei controlli lato client in Javascript, delle pagine HTML e del relativo CSS: come raggiungerci e le nostre camere.\\

Silvia si è occupata del CSS di stampa e della stesura della relazione insieme a Enrico, Francesco e Tobia. 