\section{Struttura}
\subsection{Gerarchia dei File}
All'interno della cartella principale si trovano i file delle pagine raggiungibili dagli utenti. Il progetto è stato, inizialmente, pensato secondo lo standard XHTML 1.0 Strict, in fase di realizzazione si è però deciso di utilizzare HTML5. 
Le pagine che costituiscono il sito sono:
\begin{itemize} 
\item index.php: Questa pagina è la prima con cui interagisce l'utente e contiene una piccola descrizione sulle motivazioni per cui vale la pena pernottare da noi;
\item come\_raggiungerci.php: Pagina dove vengono illustrate le possibilità con cui poterci raggiungere: aereo, treno e auto, con annessa mappa interattiva;
\item camere.php: Pagina che illustra le possibili tipologie di camere che si possono prenotare, dalla più economica a quella più lussuosa;
\item dintorni.php: In questa pagina vengono forniti spunti riguardo luoghi da visitare, come ad esempio musei o piazze, e vengono suggeriti ristoranti e bar;
\item servizi.php: Pagina contenente i servizi offerti dall'hotel, come wi-fi, parcheggio e accesso agli animali;
\item contatti.php: Pagina contenente il numero di telefono e la e-mail dell'hotel;
\item prenota.php: Pagina con il form per la prenotazione della camera.
\end{itemize}
\subsection{Scelta HTML5}
Si è deciso di utilizzare HTML5 per i seguenti motivi:
\begin{itemize}
\item permettere la validazione del codice contenente una mappa importata da Google Maps (\underline{\color{Blue}https://www.google.it/maps});
\item la possibilità di inserire un'immagine all'interno del tag anchor a;
\item permette la presenza di block elements dentro i tag a;
\item attributo \textit{placeholder}: aiuta l'utente della compilazione dei form;
\end{itemize} 
%\end{section}
