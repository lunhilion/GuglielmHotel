\section{Gestione dei Dati}
Il PHP in quetso sito è stato utilizzato per permettere la manipolazione e visualizzazione dei dati presenti nel database. Principalmente si può dividere in due categorie:
\begin{itemize}
	\item [1]. Funzioni che genereano le pagine visualizzate dai visitatori;
	\item [2]. Funzioni che permettono agli utenti e amministratori di modificare i dati presenti.
\end{itemize}
Nella seconda categoria abbiamo racchiuso sia le funzioni destinate all'utente(per prenotare la stanza), sia quella dedicata esclusivamente agli amministatori. Le funzioni amministrative sono inserite nel file cpadmin.php, mentre quelle che riguardano gli utenti che devono prenotare una stanza sul prenota.php.
\subsection{Funzioni visibili ai visitatori}
....

