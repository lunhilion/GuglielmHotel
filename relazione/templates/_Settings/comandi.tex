%Generali

\newcommand{\G}{\tiny\textit{{G}}\normalsize}

\newcommand{\RQQ}[1]{\tiny{\uppercase{\textit{#1}}}}


\newcommand{\diaryEntry}[5]{#2 & \emph{#4} & #3 & #5 & #1\\ \hline}

%comando per una nuova riga nella tabella del diario delle modifiche
\newcommand{\specialcell}[2][c]{%
	\begin{tabular}[#1]{@{}c@{}}#2\end{tabular}}

\renewcommand*\sectionmark[1]{\markboth{#1}{}}
\renewcommand*\subsectionmark[1]{\markright{#1}}


\newcommand{\myincludegraphics}[2][]{%
	\setbox0=\hbox{\phantom{X}}%
	\vtop{
		\hbox{\phantom{X}}
		\vskip-\ht0
		\hbox{\includegraphics[#1]{#2}}}}


\newcommand{\nogloxy}[1]{#1} % comando da usare per evitare di mettere il mark del 
\newcommand{\impl}{\textcolor{Green}{Implementato}}
\newcommand{\implno}{\textcolor{Red}{Non Implementato}}

% Formattazione paragrafo e sotto-paragrafo
\newcommand{\myparagraph}[1]{\paragraph{#1}\hfill\vspace{1em}\\\noindent}
\newcommand{\mysubparagraph}[1]{\subparagraph{#1}\hfill\vspace{1em}\\\noindent}


% Standard ISO
\newcommand{\iso}[1]{$[$#1$]$}

%
% COLORI
%
% Intestazione tabelle
\definecolor{I}{HTML}{003153} %{22313D}
% Righe dispari tabelle
\definecolor{D}{HTML}{F0F8FF}%{E8EAF6}
% Righe pari tabelle
\definecolor{P}{HTML}{A1CAF1}%{7DA7D9}

\lstdefinestyle{customTeX}{
	belowcaptionskip=1\baselineskip,
	breaklines=true,
	xleftmargin=\parindent,
	language=TeX,
	showstringspaces=false,
	basicstyle=\footnotesize\ttfamily,
	keywordstyle=\bfseries\color{green!40!black},
	commentstyle=\itshape\color{purple!40!black},
	identifierstyle=\color{Gray},
	stringstyle=\color{orange},
}
